\documentclass[aspectratio=169]{beamer}

\mode<presentation>
{
  \usetheme{Warsaw}
  % or ...

  \setbeamercovered{transparent}
  % or whatever (possibly just delete it)
}


\usepackage[english]{babel}
\usepackage[latin1]{inputenc}
\usepackage{graphicx}
%\usepackage{times}
%\usepackage[T1]{fontenc}
% Or whatever. Note that the encoding and the font should match. If T1
% does not look nice, try deleting the line with the fontenc.

\usepackage{amsmath,amsfonts,amssymb}

\newcommand{\pkg}{\textbf}
\newcommand{\code}{\texttt}


\title[The R Language]{Introduction to the R Language}

\subtitle{Loop Functions}

\date{Computing for Data Analysis}



\begin{document}

\begin{frame}
  \titlepage
\end{frame}

\begin{frame}{Looping on the Command Line}
Writing for, while loops is useful when programming but not
particularly easy when working interactively on the command line.
There are some functions which implement looping to make life easier.
\begin{itemize}
\item
\code{lapply}:  Loop over a list and evaluate a function on each element
\item
\code{sapply}:  Same as \code{lapply} but try to simplify the result
\item
\code{apply}:  Apply a function over the margins of an array
\item
\code{tapply}:  Apply a function over subsets of a vector
\item
\code{mapply}:  Multivariate version of \code{lapply}
\end{itemize}
An auxiliary function \code{split} is also useful, particularly in
conjunction with \code{lapply}.
\end{frame}

\begin{frame}[fragile]{lapply}
\code{lapply} takes three arguments: a list \code{X}, a function (or
the name of a function) \code{FUN}, and other arguments via its
\code{...} argument.  If \code{X} is not a list, it will be coerced to
a list using \code{as.list}.
\begin{verbatim}
> lapply
function (X, FUN, ...) 
{
    FUN <- match.fun(FUN)
    if (!is.vector(X) || is.object(X)) 
        X <- as.list(X)
    .Internal(lapply(X, FUN))
}
\end{verbatim}
The actual looping is done internally in C code.
\end{frame}


\begin{frame}[fragile]{lapply}
\code{lapply} always returns a list, regardless of the class of the
input.
\begin{verbatim}
> x <- list(a = 1:5, b = rnorm(10))
> lapply(x, mean)
$a
[1] 3

$b
[1] 0.0296824
\end{verbatim}
\end{frame}


\begin{frame}[fragile]{lapply}
\begin{verbatim}
> x <- list(a = 1:4, b = rnorm(10), c = rnorm(20, 1), d = rnorm(100, 5))
> lapply(x, mean)
$a
[1] 2.5

$b
[1] 0.06082667

$c
[1] 1.467083

$d
[1] 5.074749
\end{verbatim}
\end{frame}


\begin{frame}[fragile]{lapply}
\begin{verbatim}
> x <- 1:4
> lapply(x, runif)
[[1]]
[1] 0.2675082

[[2]]
[1] 0.2186453 0.5167968

[[3]]
[1] 0.2689506 0.1811683 0.5185761

[[4]]
[1] 0.5627829 0.1291569 0.2563676 0.7179353
\end{verbatim}
\end{frame}

\begin{frame}[fragile]{lapply}
\begin{verbatim}
> x <- 1:4
> lapply(x, runif, min = 0, max = 10)
[[1]]
[1] 3.302142

[[2]]
[1] 6.848960 7.195282

[[3]]
[1] 3.5031416 0.8465707 9.7421014

[[4]]
[1] 1.195114 3.594027 2.930794 2.766946
\end{verbatim}
\end{frame}


\begin{frame}[fragile]{lapply}
\code{lapply} and friends make heavy use of \textit{anonymous functions}.
\begin{verbatim}
> x <- list(a = matrix(1:4, 2, 2), b = matrix(1:6, 3, 2))
> x
$a
     [,1] [,2]
[1,]    1    3
[2,]    2    4

$b
     [,1] [,2]
[1,]    1    4
[2,]    2    5
[3,]    3    6
\end{verbatim}
\end{frame}


\begin{frame}[fragile]{lapply}
An anonymous function for extracting the first column of each matrix.
\begin{verbatim}
> lapply(x, function(elt) elt[,1])
$a
[1] 1 2

$b
[1] 1 2 3
\end{verbatim}
\end{frame}


\begin{frame}[fragile]{sapply}
\code{sapply} will try to simplify the result of \code{lapply} if
possible.
\begin{itemize}
\item
If the result is a list where every element is length 1, then a vector
is returned
\item
If the result is a list where every element is a vector of the same
length ($> 1$), a matrix is returned.
\item
If it can't figure things out, a list is returned
\end{itemize}
\end{frame}



\begin{frame}[fragile]{sapply}
\begin{verbatim}
> x <- list(a = 1:4, b = rnorm(10), c = rnorm(20, 1), d = rnorm(100, 5))
> lapply(x, mean)
$a
[1] 2.5

$b
[1] 0.06082667

$c
[1] 1.467083

$d
[1] 5.074749
\end{verbatim}
\end{frame}

\begin{frame}[fragile]{sapply}
\begin{verbatim}
> sapply(x, mean)
         a          b          c          d 
2.50000000 0.06082667 1.46708277 5.07474950


> mean(x)
[1] NA
Warning message:
In mean.default(x) : argument is not numeric or logical: returning NA
\end{verbatim}
\end{frame}

\begin{frame}[fragile]{apply}
\code{apply} is used to a evaluate a function (often an anonymous one)
over the margins of an array.
\begin{itemize}
\item
It is most often used to apply a function to the rows or columns of a matrix
\item
It can be used with general arrays, e.g. taking the average of an
array of matrices
\item
It is not really faster than writing a loop, but it works in one line!
\end{itemize}
\end{frame}


\begin{frame}[fragile]{apply}
\begin{verbatim}
> str(apply)
function (X, MARGIN, FUN, ...)  
\end{verbatim}
\begin{itemize}
\item
\code{X} is an array
\item
\code{MARGIN} is an integer vector indicating which margins should be
``retained''.
\item
\code{FUN} is a function to be applied
\item
\code{...} is for other arguments to be passed to \code{FUN}
\end{itemize}
\end{frame}

\begin{frame}[fragile]{apply}
\begin{verbatim}
> x <- matrix(rnorm(200), 20, 10)
> apply(x, 2, mean)
 [1]  0.04868268  0.35743615 -0.09104379
 [4] -0.05381370 -0.16552070 -0.18192493
 [7]  0.10285727  0.36519270  0.14898850
[10]  0.26767260

> apply(x, 1, sum)
 [1] -1.94843314  2.60601195  1.51772391
 [4] -2.80386816  3.73728682 -1.69371360
 [7]  0.02359932  3.91874808 -2.39902859
[10]  0.48685925 -1.77576824 -3.34016277
[13]  4.04101009  0.46515429  1.83687755
[16]  4.36744690  2.21993789  2.60983764
[19] -1.48607630  3.58709251
\end{verbatim}
\end{frame}


\begin{frame}{col/row sums and means}
For sums and means of matrix dimensions, we have some shortcuts.
\begin{itemize}
\item
\code{rowSums} = apply(x, 1, sum)
\item
\code{rowMeans} = apply(x, 1, mean)
\item
\code{colSums} = apply(x, 2, sum)
\item
\code{colMeans} = apply(x, 2, mean)
\end{itemize}
The shortcut functions are \textit{much} faster, but you won't notice
unless you're using a large matrix.
\end{frame}

\begin{frame}[fragile]{Other Ways to Apply}
Quantiles of the rows of a matrix.
\begin{verbatim}
> x <- matrix(rnorm(200), 20, 10)
> apply(x, 1, quantile, probs = c(0.25, 0.75))
          [,1]        [,2]       [,3]        [,4]
25% -0.3304284 -0.99812467 -0.9186279 -0.49711686
75%  0.9258157  0.07065724  0.3050407 -0.06585436
           [,5]       [,6]      [,7]       [,8]
25% -0.05999553 -0.6588380 -0.653250 0.01749997
75%  0.52928743  0.3727449  1.255089 0.72318419
          [,9]      [,10]      [,11]      [,12]
25% -1.2467955 -0.8378429 -1.0488430 -0.7054902
75%  0.3352377  0.7297176  0.3113434  0.4581150
         [,13]      [,14]      [,15]      [,16]
25% -0.1895108 -0.5729407 -0.5968578 -0.9517069
75%  0.5326299  0.5064267  0.4933852  0.8868922
         [,17]      [,18]      [,19]     [,20]
25% -0.2502935 -0.7488003 -0.7190923 -0.638243
75%  0.7763024  0.2873202  0.6416363  1.271602
\end{verbatim}
\end{frame}


\begin{frame}[fragile]{apply}
Average matrix in an array
\begin{verbatim}
> a <- array(rnorm(2 * 2 * 10), c(2, 2, 10))
> apply(a, c(1, 2), mean)
           [,1]        [,2]
[1,] -0.2353245 -0.03980211
[2,] -0.3339748  0.04364908

> rowMeans(a, dims = 2)
           [,1]        [,2]
[1,] -0.2353245 -0.03980211
[2,] -0.3339748  0.04364908
\end{verbatim}
\end{frame}

\begin{frame}[fragile]{tapply}
\code{tapply} is used to apply a function over subsets of a vector.  I
don't know why it's called \code{tapply}.
\begin{verbatim}
> str(tapply)
function (X, INDEX, FUN = NULL, ..., simplify = TRUE)  
\end{verbatim}
\begin{itemize}
\item
\code{X} is a vector
\item
\code{INDEX} is a factor or a list of factors (or else they are coerced to
factors)
\item
\code{FUN} is a function to be applied
\item
\code{...} contains other arguments to be passed \code{FUN}
\item
\code{simplify}, should we simplify the result?
\end{itemize}
\end{frame}

\begin{frame}[fragile]{tapply}
Take group means.
\begin{verbatim}
> x <- c(rnorm(10), runif(10), rnorm(10, 1))
> f <- gl(3, 10)
> f
 [1] 1 1 1 1 1 1 1 1 1 1 2 2 2 2 2 2 2 2 2 2 3 3 3
[24] 3 3 3 3 3 3 3
Levels: 1 2 3
> tapply(x, f, mean)
        1         2         3 
0.1144464 0.5163468 1.2463678 
\end{verbatim}
\end{frame}

\begin{frame}[fragile]{tapply}
Take group means without simplification.
\begin{verbatim}
> tapply(x, f, mean, simplify = FALSE)
$`1`
[1] 0.1144464

$`2`
[1] 0.5163468

$`3`
[1] 1.246368
\end{verbatim}
\end{frame}


\begin{frame}[fragile]{tapply}
Find group ranges.
\begin{verbatim}
> tapply(x, f, range)
$`1`
[1] -1.097309  2.694970

$`2`
[1] 0.09479023 0.79107293

$`3`
[1] 0.4717443 2.5887025
\end{verbatim}
\end{frame}


\begin{frame}[fragile]{split}
\code{split} takes a vector or other objects and splits it into groups
determined by a factor or list of factors.
\begin{verbatim}
> str(split)
function (x, f, drop = FALSE, ...)  
\end{verbatim}
\begin{itemize}
\item
\code{x} is a vector (or list) or data frame
\item
\code{f} is a factor (or coerced to one) or a list of factors
\item
\code{drop} indicates whether empty factors levels should be dropped
\end{itemize}
\end{frame}

\begin{frame}[fragile]{split}
\begin{verbatim}
> x <- c(rnorm(10), runif(10), rnorm(10, 1))
> f <- gl(3, 10)
> split(x, f)
$`1`
 [1] -0.8493038 -0.5699717 -0.8385255 -0.8842019
 [5]  0.2849881  0.9383361 -1.0973089  2.6949703
 [9]  1.5976789 -0.1321970

$`2`
 [1] 0.09479023 0.79107293 0.45857419 0.74849293
 [5] 0.34936491 0.35842084 0.78541705 0.57732081
 [9] 0.46817559 0.53183823

$`3`
 [1] 0.6795651 0.9293171 1.0318103 0.4717443
 [5] 2.5887025 1.5975774 1.3246333 1.4372701
 [9] 1.3961579 1.0068999
\end{verbatim}
\end{frame}

\begin{frame}[fragile]{split}
A common idiom is \code{split} followed by an \code{lapply}.
\begin{verbatim}
> lapply(split(x, f), mean)
$`1`
[1] 0.1144464

$`2`
[1] 0.5163468

$`3`
[1] 1.246368
\end{verbatim}
\end{frame}


\begin{frame}[fragile]{Splitting a Data Frame}
\begin{verbatim}
> library(datasets)
> head(airquality)
  Ozone Solar.R Wind Temp Month Day
1    41     190  7.4   67     5   1
2    36     118  8.0   72     5   2
3    12     149 12.6   74     5   3
4    18     313 11.5   62     5   4
5    NA      NA 14.3   56     5   5
6    28      NA 14.9   66     5   6
\end{verbatim}
\end{frame}

\begin{frame}[fragile]{Splitting a Data Frame}
\begin{verbatim}
> s <- split(airquality, airquality$Month)
> lapply(s, function(x) colMeans(x[, c("Ozone", "Solar.R", "Wind")]))
$`5`
   Ozone  Solar.R     Wind 
      NA       NA 11.62258 

$`6`
    Ozone   Solar.R      Wind 
       NA 190.16667  10.26667 

$`7`
     Ozone    Solar.R       Wind 
        NA 216.483871   8.941935 

$`8`
   Ozone  Solar.R     Wind 
      NA       NA 8.793548 

$`9`
   Ozone  Solar.R     Wind 
      NA 167.4333  10.1800 
\end{verbatim}
\end{frame}


\begin{frame}[fragile]{Splitting a Data Frame}
\begin{verbatim}
> sapply(s, function(x) colMeans(x[, c("Ozone", "Solar.R", "Wind")]))
               5         6          7        8        9
Ozone         NA        NA         NA       NA       NA
Solar.R       NA 190.16667 216.483871       NA 167.4333
Wind    11.62258  10.26667   8.941935 8.793548  10.1800



> sapply(s, function(x) colMeans(x[, c("Ozone", "Solar.R", "Wind")], 
                                 na.rm = TRUE))
                5         6          7          8         9
Ozone    23.61538  29.44444  59.115385  59.961538  31.44828
Solar.R 181.29630 190.16667 216.483871 171.857143 167.43333
Wind     11.62258  10.26667   8.941935   8.793548  10.18000
\end{verbatim}
\end{frame}



\begin{frame}[fragile]{Splitting on More than One Level}
\begin{verbatim}
> x <- rnorm(10)
> f1 <- gl(2, 5)
> f2 <- gl(5, 2)
> f1
 [1] 1 1 1 1 1 2 2 2 2 2
Levels: 1 2
> f2
 [1] 1 1 2 2 3 3 4 4 5 5
Levels: 1 2 3 4 5
> interaction(f1, f2)
 [1] 1.1 1.1 1.2 1.2 1.3 2.3 2.4 2.4 2.5 2.5
10 Levels: 1.1 2.1 1.2 2.2 1.3 2.3 1.4 ... 2.5
\end{verbatim}
\end{frame}

\begin{frame}[fragile]{Splitting on More than One Level}
Interactions can create empty levels.
\begin{verbatim}
> str(split(x, list(f1, f2)))
List of 10
 $ 1.1: num [1:2] -0.378  0.445
 $ 2.1: num(0) 
 $ 1.2: num [1:2] 1.4066 0.0166
 $ 2.2: num(0) 
 $ 1.3: num -0.355
 $ 2.3: num 0.315
 $ 1.4: num(0) 
 $ 2.4: num [1:2] -0.907  0.723
 $ 1.5: num(0) 
 $ 2.5: num [1:2] 0.732 0.360
\end{verbatim}
\end{frame}

\begin{frame}[fragile]{split}
Empty levels can be dropped.
\begin{verbatim}
> str(split(x, list(f1, f2), drop = TRUE))
List of 6
 $ 1.1: num [1:2] -0.378  0.445
 $ 1.2: num [1:2] 1.4066 0.0166
 $ 1.3: num -0.355
 $ 2.3: num 0.315
 $ 2.4: num [1:2] -0.907  0.723
 $ 2.5: num [1:2] 0.732 0.360
\end{verbatim}
\end{frame}




\begin{frame}[fragile]{mapply}
\code{mapply} is a multivariate apply of sorts which applies a
function in parallel over a set of arguments.
\begin{verbatim}
> str(mapply)
function (FUN, ..., MoreArgs = NULL, SIMPLIFY = TRUE, 
          USE.NAMES = TRUE)
\end{verbatim}
\begin{itemize}
\item
\code{FUN} is a function to apply
\item
\code{...} contains arguments to apply over
\item
\code{MoreArgs} is a list of other arguments to \code{FUN}.
\item
\code{SIMPLIFY} indicates whether the result should be simplified
\end{itemize}
\end{frame}


\begin{frame}[fragile]{mapply}
The following is tedious to type
\begin{verbatim}
list(rep(1, 4), rep(2, 3), rep(3, 2), rep(4, 1))
\end{verbatim}
Instead we can do
\begin{verbatim}
> mapply(rep, 1:4, 4:1)
[[1]]
[1] 1 1 1 1

[[2]]
[1] 2 2 2

[[3]]
[1] 3 3

[[4]]
[1] 4
\end{verbatim}
\end{frame}


\begin{frame}[fragile]{Vectorizing a Function}
\begin{verbatim}
> noise <- function(n, mean, sd) {
+         rnorm(n, mean, sd)
+ }
> noise(5, 1, 2)
[1]  2.4831198  2.4790100  0.4855190 -1.2117759
[5] -0.2743532

> noise(1:5, 1:5, 2)
[1] -4.2128648 -0.3989266  4.2507057  1.1572738
[5]  3.7413584
\end{verbatim}
\end{frame}

\begin{frame}[fragile]{Instant Vectorization}
\begin{verbatim}
> mapply(noise, 1:5, 1:5, 2)
[[1]]
[1] 1.037658

[[2]]
[1] 0.7113482 2.7555797

[[3]]
[1] 2.769527 1.643568 4.597882

[[4]]
[1] 4.476741 5.658653 3.962813 1.204284

[[5]]
[1] 4.797123 6.314616 4.969892 6.530432 6.723254
\end{verbatim}
\end{frame}

\begin{frame}[fragile]{Instant Vectorization}
Which is the same as
\begin{verbatim}
list(noise(1, 1, 2), noise(2, 2, 2),
     noise(3, 3, 2), noise(4, 4, 2),
     noise(5, 5, 2))
\end{verbatim}
\end{frame}


\end{document}
