\documentclass[aspectratio=169]{beamer}

\mode<presentation>
{
  \usetheme{Warsaw}
  % or ...

  \setbeamercovered{transparent}
  % or whatever (possibly just delete it)
}


\usepackage[english]{babel}
\usepackage[latin1]{inputenc}
\usepackage{graphicx}
%\usepackage{times}
%\usepackage[T1]{fontenc}
% Or whatever. Note that the encoding and the font should match. If T1
% does not look nice, try deleting the line with the fontenc.

\usepackage{amsmath,amsfonts,amssymb}

\newcommand{\pkg}{\textbf}
\newcommand{\code}{\texttt}


\title[The R Language]{Introduction to the R Language}

\subtitle{Data Types and Basic Operations}

%\author{Roger D. Peng}
% - Give the names in the same order as the appear in the paper.
% - Use the \inst{?} command only if the authors have different
%   affiliation.

%\institute{
%  \inst{1}%
%  Department Biostatistics\\
%  Johns Hopkins Bloomberg School of Public Health
%  \and
%  \inst{2}%
%  Department of Preventive Medicine\\
%  Feinberg School of Medicine, Northwestern University
%}
% - Use the \inst command only if there are several affiliations.
% - Keep it simple, no one is interested in your street address.

\date{Computing for Data Analysis}

\setbeamertemplate{footline}[page number]


\begin{document}

\begin{frame}
  \titlepage
\end{frame}








\begin{frame}[fragile]{Subsetting}
There are a number of operators that can be used to extract subsets of
R objects.
\begin{itemize}
\item
\verb+[+ always returns an object of the same class as the original;
can be used to select more than one element (there is one exception)
\item
\verb+[[+ is used to extract elements of a list or a data frame; it
can only be used to extract a single element and the class of the
returned object will not necessarily be a list or data frame
\item
\verb+$+ is used to extract elements of a list or data frame by name;
semantics are similar to hat of \verb+[[+.
\end{itemize}
\end{frame}

\begin{frame}[fragile]{Subsetting}
\begin{verbatim}
> x <- c("a", "b", "c", "c", "d", "a")
> x[1]
[1] "a"
> x[2]
[1] "b"
> x[1:4]
[1] "a" "b" "c" "c"
> x[x > "a"]
[1] "b" "c" "c" "d"
> u <- x > "a"
> u
[1] FALSE  TRUE  TRUE  TRUE  TRUE FALSE
> x[u]
[1] "b" "c" "c" "d"
\end{verbatim}
\end{frame}

\begin{frame}[fragile]{Subsetting a Matrix}
Matrices can be subsetted in the usual way with $(i, j)$ type indices.
\begin{verbatim}
> x <- matrix(1:6, 2, 3)
> x[1, 2]
[1] 3
> x[2, 1]
[1] 2
\end{verbatim}
Indices can also be missing.
\begin{verbatim}
> x[1, ]
[1] 1 3 5
> x[, 2]
[1] 3 4
\end{verbatim}
\end{frame}

\begin{frame}[fragile]{Subsetting a Matrix}
By default, when a single element of a matrix is retrieved, it is
returned as a vector of length 1 rather than a $1\times 1$ matrix.
This behavior can be turned off by setting \code{drop = FALSE}.
\begin{verbatim}
> x <- matrix(1:6, 2, 3)
> x[1, 2]
[1] 3

> x[1, 2, drop = FALSE]
     [,1]
[1,]    3
\end{verbatim}
\end{frame}

\begin{frame}[fragile]{Subsetting a Matrix}
Similarly, subsetting a single column or a single row will give you a
vector, not a matrix (by default).
\begin{verbatim}
> x <- matrix(1:6, 2, 3)
> x[1, ]
[1] 1 3 5
> x[1, , drop = FALSE]
     [,1] [,2] [,3]
[1,]    1    3    5
\end{verbatim}
\end{frame}

\begin{frame}[fragile]{Subsetting Lists}
\begin{verbatim}
> x <- list(foo = 1:4, bar = 0.6)
> x[1]
$foo
[1] 1 2 3 4

> x[[1]]
[1] 1 2 3 4

> x$bar
[1] 0.6
> x[["bar"]]
[1] 0.6
> x["bar"]
$bar
[1] 0.6
\end{verbatim}
\end{frame}

\begin{frame}[fragile]{Subsetting Lists}
Extracting multiple elements of a list.
\begin{verbatim}
> x <- list(foo = 1:4, bar = 0.6, baz = "hello")
> x[c(1, 3)]
$foo
[1] 1 2 3 4

$baz
[1] "hello"
\end{verbatim}
\end{frame}

\begin{frame}[fragile]{Subsetting Lists}
The \verb+[[+ operator can be used with \textit{computed} indices;
\verb+$+ can only be used with literal names.
\begin{verbatim}
> x <- list(foo = 1:4, bar = 0.6, baz = "hello")
> name <- "foo"
> x[[name]]  ## computed index for `foo'
[1] 1 2 3 4
> x$name     ## element `name' doesn't exist!
NULL
> x$foo
[1] 1 2 3 4  ## element `foo' does exist
\end{verbatim}
\end{frame}

\begin{frame}[fragile]{Subsetting Nested Elements of a List}
The \verb+[[+ can take an integer sequence.
\begin{verbatim}
> x <- list(a = list(10, 12, 14), b = c(3.14, 2.81))
> x[[c(1, 3)]]
[1] 14
> x[[1]][[3]]
[1] 14

> x[[c(2, 1)]]
[1] 3.14
\end{verbatim}
\end{frame}

\begin{frame}[fragile]{Partial Matching}
Partial matching of names is allowed with \verb+[[+ and \verb+$+.
\begin{verbatim}
> x <- list(aardvark = 1:5)
> x$a
[1] 1 2 3 4 5
> x[["a"]]
NULL
> x[["a", exact = FALSE]]
[1] 1 2 3 4 5
\end{verbatim}
\end{frame}

\begin{frame}[fragile]{Removing NA Values}
A common task is to remove missing values (\code{NA}s).
\begin{verbatim}
> x <- c(1, 2, NA, 4, NA, 5)
> bad <- is.na(x)
> x[!bad]
[1] 1 2 4 5
\end{verbatim}
\end{frame}

\begin{frame}[fragile]{Removing NA Values}
What if there are multiple things and you want to take the subset with
no missing values?
\begin{verbatim}
> x <- c(1, 2, NA, 4, NA, 5)
> y <- c("a", "b", NA, "d", NA, "f")
> good <- complete.cases(x, y)
> good
[1]  TRUE  TRUE FALSE  TRUE FALSE  TRUE
> x[good]
[1] 1 2 4 5
> y[good]
[1] "a" "b" "d" "f"
\end{verbatim}
\end{frame}

\begin{frame}[fragile]{Removing NA Values}
\begin{verbatim}
> airquality[1:6, ]
  Ozone Solar.R Wind Temp Month Day
1    41     190  7.4   67     5   1
2    36     118  8.0   72     5   2
3    12     149 12.6   74     5   3
4    18     313 11.5   62     5   4
5    NA      NA 14.3   56     5   5
6    28      NA 14.9   66     5   6
> good <- complete.cases(airquality)
> airquality[good, ][1:6, ]
  Ozone Solar.R Wind Temp Month Day
1    41     190  7.4   67     5   1
2    36     118  8.0   72     5   2
3    12     149 12.6   74     5   3
4    18     313 11.5   62     5   4
7    23     299  8.6   65     5   7
8    19      99 13.8   59     5   8
\end{verbatim}
\end{frame}


\end{document}
