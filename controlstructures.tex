\documentclass[aspectratio=169]{beamer}

\mode<presentation>
{
  \usetheme{Warsaw}
  % or ...

  \setbeamercovered{transparent}
  % or whatever (possibly just delete it)
}


\usepackage[english]{babel}
\usepackage[latin1]{inputenc}
\usepackage{graphicx}
%\usepackage{times}
%\usepackage[T1]{fontenc}
% Or whatever. Note that the encoding and the font should match. If T1
% does not look nice, try deleting the line with the fontenc.

\usepackage{amsmath,amsfonts,amssymb}

\newcommand{\pkg}{\textbf}
\newcommand{\code}{\texttt}


\title[The R Language]{Introduction to the R Language}

\subtitle{Control Structures}

\date{Computing for Data Analysis}

\setbeamertemplate{footline}[page number]




\begin{document}

\begin{frame}
  \titlepage
\end{frame}

\begin{frame}{Control Structures}
Control structures in R allow you to control the flow of execution of
the program, depending on runtime conditions.  Common structures are
\begin{itemize}
\item
\code{if}, \code{else}: testing a condition
\item
\code{for}: execute a loop a fixed number of times
\item
\code{while}: execute a loop \textit{while} a condition is true
\item
\code{repeat}:  execute an infinite loop
\item
\code{break}: break the execution of a loop
\item
\code{next}: skip an interation of a loop
\item
\code{return}: exit a function
\end{itemize}
Most control structures are not used in interactive sessions, but
rather when writing functions or longer expresisons.
\end{frame}


\begin{frame}[fragile]{Control Structures: if}
\begin{verbatim}
if(<condition>) {
        ## do something
} else {
        ## do something else
}

if(<condition1>) {
        ## do something
} else if(<condition2>)  {
        ## do something different
} else {
        ## do something different
}
\end{verbatim}
\end{frame}

\begin{frame}[fragile]{if}
This is a valid if/else structure.
\begin{verbatim}
if(x > 3) {
        y <- 10
} else {
        y <- 0
}
\end{verbatim}
So is this one.
\begin{verbatim}
y <- if(x > 3) {
        10
} else {
        0
}
\end{verbatim}
\end{frame}

\begin{frame}[fragile]{if}
Of course, the \code{else} clause is not necessary.
\begin{verbatim}
if(<condition1>) {

}

if(<condition2>) {

}
\end{verbatim}
\end{frame}


\begin{frame}[fragile]{for}
\code{for} loops take an interator variable and assign it successive
values from a sequence or vector.  For loops are most commonly used
for iterating over the elements of an object (list, vector, etc.)
\begin{verbatim}
for(i in 1:10) {
        print(i)
}
\end{verbatim}
This loop takes the \code{i} variable and in each iteration of the
loop gives it values 1, 2, 3, ..., 10, and then exits.
\end{frame}

\begin{frame}[fragile]{for}
These three loops have the same behavior.
\begin{verbatim}
x <- c("a", "b", "c", "d")

for(i in 1:4) {
        print(x[i])
}

for(i in seq_along(x)) {
        print(x[i])
}

for(letter in x) {
        print(letter)
}

for(i in 1:4) print(x[i])
\end{verbatim}
\end{frame}

\begin{frame}[fragile]{Nested for loops}
\code{for} loops can be nested.
\begin{verbatim}
x <- matrix(1:6, 2, 3)

for(i in seq_len(nrow(x))) {
        for(j in seq_len(ncol(x))) {
                print(x[i, j])
        }
}
\end{verbatim}
Be careful with nesting though.  Nesting beyond 2--3 levels is often
very difficult to read/understand.
\end{frame}

\begin{frame}[fragile]{while}
While loops begin by testing a condition.  If it is true, then they
execute the loop body.  Once the loop body is executed, the condition
is tested again, and so forth.
\begin{verbatim}
count <- 0

while(count < 10) {
        print(count)
        count <- count + 1
}
\end{verbatim}
While loops can potentially result in infinite loops if not written
properly.  Use with care!
\end{frame}

\begin{frame}[fragile]{while}
Sometimes there will be more than one condition in the test.
\begin{verbatim}
z <- 5

while(z >= 3 && z <= 10) {
        print(z)
        coin <- rbinom(1, 1, 0.5)

        if(coin == 1) {  ## random walk
                z <- z + 1
        } else {
                z <- z - 1
        }
}
\end{verbatim}
Conditions are always evaluated from left to right.
\end{frame}

\begin{frame}[fragile]{repeat}
Repeat initiates an infinite loop; these are not commonly used in
statistical applications but they do have their uses.  The only way to
exit a \code{repeat} loop is to call \code{break}.
\begin{verbatim}
x0 <- 1
tol <- 1e-8

repeat {
        x1 <- computeEstimate()

        if(abs(x1 - x0) < tol) {
                break
        } else {
                x0 <- x1
        }
}
\end{verbatim}
\end{frame}

\begin{frame}{repeat}
The loop in the previous slide is a bit dangerous because there's no
guarantee it will stop.  Better to set a hard limit on the number of
iterations (e.g. using a for loop) and then report whether convergence
was achieved or not.
\end{frame}

\begin{frame}[fragile]{next, return}
\code{next} is used to skip an iteration of a loop
\begin{verbatim}
for(i in 1:100) {
        if(i <= 20) {
                ## Skip the first 20 iterations
                next
        }
        ## Do something here
}
\end{verbatim}
\code{return} signals that a function should exit and return a given
value
\end{frame}


\begin{frame}{Control Structures}
Summary
\begin{itemize}
  \item Control structures like \code{if}, \code{while}, and \code{for}
    allow you to control the flow of an R program
  \item Infinite loops should generally be avoided, even if they are
    theoretically correct.
  \item Control structures mentiond here are primarily useful for
    writing programs; for command-line interactive work, the *apply
    functions are more useful.
\end{itemize}
\end{frame}




\end{document}


